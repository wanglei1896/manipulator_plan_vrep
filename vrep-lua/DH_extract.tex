\documentclass{article}
\usepackage{CJK}
\begin{document}
\begin{CJK}{UTF8}{song}
\title{从任意变换矩阵中提取D-H参数}
\maketitle
\textbf{1.任意变换矩阵的分解}\\
任意的变换矩阵可以分解为三个子矩阵的乘积:$T_1T_2T_3=T$。
若用$T_x$表示沿x轴的旋转平移变换,$T_y$表示沿y轴的旋转平移变换,$T_z$表示沿z轴的旋转平移变换,
则根据欧拉旋转定理,三个子矩阵中相邻两矩阵不绕同一轴旋转平移。
如以下变换可表示任意旋转变换:\\
$T=T_xT_yT_z$,\qquad$T=T_xT_yT_x$,\qquad$T=T_zT_xT_z$,\qquad$T=T_zT_yT_x$\\
以下变换则不能表示任意旋转变换:\\
$T=T_xT_yT_y=T_xT_y$,\qquad$T=T_xT_xT_x=T_x$,\qquad$T=T_xT_xT_z=T_xT_z$\\
\\
\textbf{2.转化为DH变换矩阵的核心方法}\\
DH参数表示的变换称为DH变换:$T_{DH}=T_zT_x$,其中,$T_z$和$T_x$分别用旋转平移参数($\theta$,$d$)以及($\alpha$,$a$)来参数化。
由前文可知,由于DH变换缺少一个沿y轴或z轴的变换,其不能表示任意变换,因此我们在这里不上一个虚拟的子矩阵$T_z^\prime$,其用($\theta^\prime$,$d^\prime$)来参数化,则任意变换可表示为$T=T_{DH}T_z^\prime$。\\
机械臂是由一个个关节连接而成的,相邻关节的坐标系之间可能存在任意变换,其运动学如下所示:\\
$$T_e=T_1\cdot T_2\cdot T_3 ... T_n$$
$$=(T_{DH_1}T_{z_1}^\prime)(T_{DH_2}T_{z_2}^\prime) ... (T_{DH_n}T_{z_n}^\prime)$$
其中$T_e$表示到末端执行器的变换,$T_i$表示第$i$到第$i+1$号关节的变换,改变括号的结合方式后,我们有
$$T_e=T_{DH_1}(T_{z_1}^\prime T_{DH_2})(T_{z_2}^\prime T_{DH3}) ... (T_{z_{n-1}}^\prime T_{DH_n})T_{z_n}^\prime$$
其中
$$T_{z_i}^\prime T_{DH_{i+1}}=T_{z_i}^\prime T_{z_{i+1}}T_{x_{i+1}}=T_{z_{i+1}}^{\prime\prime} T_{x_{i+1}}=T_{DH_{i+1}}^\prime$$
于是
$$T_e=T_{DH_1}T_{DH_2}^\prime ... T_{DH_n}^\prime T_{z_n}^\prime$$
由于$T_n$一般仅仅是一个$T_z$变换,所以可忽略$T_{z_n}^\prime$
最后
$$T_e=T_{DH_1}T_{DH_2}^\prime T_{DH_3}^\prime ... T_{DH_n}^\prime$$
其中$T_{DH_1}$用($\theta_1$,$d$,$\alpha_1$,$a_1$)参数化,$T_{DH_{i+1}}^\prime$用($\theta_i^\prime+\theta_{i+1}$,$d_i^\prime+d_{i+1}$,$\alpha_{i+1}$,$a_{i+1}$)参数化($i \in [1,n)$)\\
\\
\textbf{3.如何从$T_i$中提取参数}($\theta_i$,$d_i$,$\alpha_i$,$a_i$,$\theta^\prime_i$,$d^\prime_i$)\\
$$T=T_zT_xT_z^\prime$$
$$={
\left[ \begin{array}{cccc}
         cos(\theta) & -sin(\theta) & 0 & 0 \\
         sin(\theta) & cos(\theta) & 0 & 0 \\
         0 & 0 & 1 & d \\
         0 & 0 & 0 & 1
       \end{array}
\right ]\cdot
\left[ \begin{array}{cccc}
         0 & 0 & 0 & a \\
         0 & cos(\alpha) & -sin(\alpha) & 0 \\
         0 & sin(\alpha) & cos(\alpha) & 0 \\
         0 & 0 & 0 & 1
       \end{array}
\right ]\cdot
\left[ \begin{array}{cccc}
         cos(\theta^\prime) & -sin(\theta^\prime) & 0 & 0 \\
         sin(\theta^\prime) & cos(\theta^\prime) & 0 & 0 \\
         0 & 0 & 1 & d^\prime \\
         0 & 0 & 0 & 1
       \end{array}
\right ]
}
$$
$$
={
\left[ \begin{array}{cccc}
         \cos\theta\cos\theta^\prime-\sin\theta\sin\theta^\prime\cos\alpha & -\sin\theta^\prime\cos\theta+\cos\theta^\prime\sin\theta\cos\alpha & \sin\theta\sin\alpha & d^\prime\sin\theta\sin\alpha+a\cos\theta \\
         \sin\theta\cos\theta^\prime+\cos\theta\sin\theta^\prime\cos\alpha & -\sin\theta^\prime\sin\theta+\cos\theta^\prime\cos\theta\cos\alpha & -\cos\theta\sin\alpha & -d^\prime\cos\theta\sin\alpha+a\sin\theta \\
         \sin\theta^\prime\sin\alpha & \cos\theta^\prime\sin\alpha & \cos\alpha & d^\prime\cos\alpha+d \\
         0 & 0 & 0 & 1
       \end{array}
\right]
}
$$
假设已知的变换矩阵为:
$$
DHT={
\left[ \begin{array}{cccc}
         T[1]& T[2] & T[3] & T[4] \\
         T[5]& T[6] & T[7] & T[8] \\
         T[9]& T[10] & T[11] & T[12] \\
         0 & 0 & 0 & 1
       \end{array}
\right]
}
$$
则先由$\cos^{-1}(T[11])$可求得$\alpha$。
若$\sin\alpha\neq0$,则结合$T[3],T[7],T[9],T[10]$可求得$\theta$与$\theta^\prime$,然后将$\theta$和$\alpha$的值代入$T[4],T[8],T[12]$,联立解三元一次方程组,即可求得$d$,$a$以及$d^\prime$;
若$\sin\alpha=0$,则矩阵可简化为:
$$
\left[ \begin{array}{cccc}
         \cos(\theta\pm\theta^\prime) & \sin(\theta\pm\theta^\prime) & 0 & a\cos\theta \\
         \sin(\theta\pm\theta^\prime) & \cos(\theta\pm\theta^\prime) & 0 & a\sin\theta \\
         0 & 0 & \pm 1 & \pm d^\prime+d \\
         0 & 0 & 0 & 1
       \end{array}
\right]
$$
由于这里$\theta$和$\theta^\prime$作用相同,我们可以令$\theta^\prime=0$,这样,其余各参数也都可轻松求出了。\\
\\
求解时要注意的点:
\begin{enumerate}
\item 仅用反三角函数求解时得到的角度值可以有多个,如$\sin\theta=0$时$\theta$可以是0也可以是$\pi$,需结合$\cos\theta$的值进一步确定;
\item 由于是浮点运算,所以比较时不能用等号,求反三角函数时因为精度问题也可能超过$[-1,1]$的取值范围,另外也得避免除以0的情况;
\end{enumerate}


\textbf{感想:}\\
本来以为这种提取DH参数的工具网上应该挺好找的,然而出乎意料竟然没找到。
vrep上本来有一个DH extractor工具,但是有各种问题,得到的DH参数我用着根本不对,也不知道是不是我没用对。
刚好现在的任务里面要适配各种类型机械臂,自动提取DH参数就显得比较重要了,于是决定自己写一个这样的工具。凭着自己的理解,大致是写出来了,然而中间确走了相当多的坑。基本思路还挺简单的,大概当天就有头绪了,但是实现上却花了很多时间,算了一下,从周四开始到周日上午最后完成,用了4天时间,这四天时间里后面三天都是在修改调bug!一个原因是代码里面条件挺多的,本身要正确实现就有点难,但最主要的障碍还是整个环境难以调试,找到问题的出处要花上很久。接下来要加强自己在代码正确实现和调试上的功力了。\\
总的来说,写完了还是有点成就感的,但毕竟花了太多时间,而且这仅仅是任务中一个为了方便点而做的工具,不知道这个投入产出比值不值。难得搞了这么长时间走了这么多坑,还是记录一下来的好。接下来就可以更方便的开展规划算法的适配工作了,还有就是不能保证现在的代码一定没问题了,毕竟我只试了4个机械臂。。。
\end{CJK}
\end{document}